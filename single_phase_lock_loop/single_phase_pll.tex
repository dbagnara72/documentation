%\documentclass[11pt,a4paper]{article}
\documentclass[11pt,a4paper]{scrartcl}
%\documentclass[11pt,a4paper,oneside]{book}
\usepackage[british,UKenglish,USenglish,english,american]{babel}
%\usepackage[a4paper, total={16cm, 23cm}]{geometry}
\usepackage[tmargin = 1in,bmargin = 1in,lmargin = 0.75in,rmargin = 
0.75in]{geometry}
\usepackage{tikz}
\usepackage{graphicx}
\usepackage{pgfplots}
\pgfplotsset{width=12cm,compat=1.9}
\usepackage{xcolor}

%\usepackage{esint}
\usepackage{tabularray}
\usepackage{makeidx}
\usepackage{epstopdf}
\usepackage{amssymb}
\usepackage{mathrsfs}

\usepackage{bm}
\usepackage{amsmath}
\usepackage{enumitem}
\usepackage[english]{varioref}
\usepackage[english]{babel}
\usepackage{lipsum}
\usepackage{fancyhdr}
\pagestyle{fancy} 
\usepackage{float}
\usepackage{empheq}
\usepackage[framemethod=tikz]{mdframed}
\usepackage{epstopdf}
\numberwithin{equation}{section}
\usepackage{eso-pic}
\usepackage{calc}
\usepackage{nccmath}
\usepackage{caption}
\usepackage{subcaption}
\usepackage{gensymb}
\usepackage{amsfonts,amsthm,epsfig,epstopdf,titling,url,array}
\usepackage{siunitx}
\sisetup{input-digits = 0123456789\pi}
\usepackage[symbol]{footmisc}
\usepackage{multicol}
\usepackage{boondox-cal}
\DeclareSIUnit\atm{atm}
\setcounter{secnumdepth}{3}
\setcounter{tocdepth}{3}
\usepackage{booktabs}
\usepackage{blindtext}
\usepackage{changepage}
% \usepackage{draftwatermark}
% \SetWatermarkText{DRAFT}
% \SetWatermarkScale{5}

\DeclareSIUnit\atm{atm}

\pagestyle{fancy} 
\fancypagestyle{firstpage}{
\rhead{
%	\begin{picture}(0,0) 
%			\put(-30,0){\includegraphics[width=1cm]{figures/MCI_4C_bw.eps}} 
%	\end{picture}
}
}
\fancyhead[L]{\slshape\nouppercase{\leftmark}}
\chead{}
\rhead{
%	\begin{picture}(0,0) 
%		\put(-30,0){\includegraphics[width=1cm]{figures/MCI_4C_bw.eps}} 
%	\end{picture}
}
\lfoot{\textit{}}
\cfoot{-\ \thepage\ -}
\rfoot{\textit{}}

\DeclareMathOperator{\rank}{rank}
\DeclareMathOperator{\atantwo}{atan2}
\DeclareMathOperator{\spn}{span}

\renewcommand{\headrulewidth}{0.4pt}
\renewcommand{\footrulewidth}{0.4pt}
\newcommand{\abs}[1]{\left|#1\right|}
\definecolor{mycolor1}{rgb}{0.97, 0.97, 0.97}
\definecolor{mycolor2}{rgb}{0.97, 0.97, 0.97}
\definecolor{tableShade}{gray}{0.9}
\newcommand{\sign}{\text{sign}}
\newcommand{\centered}[1]{\begin{tabular}{@{}l@{}} #1 \end{tabular}}
\theoremstyle{it}

\newtheorem{defn}{Definition}[section]
\newtheorem{assumption}{Assumption}[section]
\newtheorem{thm}{Theorem}[section]
\newtheorem{lemma}{Lemma}[section]
\newtheorem{corollary}{Corollary}[section]


%\newtheorem{defn}{Definition}[chapter]
%\newtheorem{assumption}{Assumption}[chapter]
%\newtheorem{thm}{Theorem}[chapter]
%\newtheorem{lemma}{Lemma}[chapter]
%\newtheorem{corollary}{Corollary}[chapter]
\theoremstyle{definition}
%\theoremstyle{it}
\newtheorem{example}{Example}[section]

\newenvironment{myitemize_1}
{ \begin{itemize}[topsep=0pt]
		\setlength{\topsep}{2pt}		
		\setlength{\itemsep}{2pt}
		\setlength{\parskip}{2pt}
		\setlength{\parsep}{2pt}     }
	{ \end{itemize}                  }

\newmdenv[innerlinewidth=0.5pt, roundcorner=4pt,backgroundcolor=mycolor2, 
linecolor=mycolor1,innerleftmargin=6pt,
innerrightmargin=6pt,innertopmargin=6pt,innerbottommargin=6pt]{mybox}

\title{\textbf{ 
	\begin{LARGE}
		Single Phase Lock Loop
	\end{LARGE}}
}
\author{\textbf{davide bagnara}}

\begin{document}
	\thispagestyle{firstpage}
	\begin{mybox}
		\maketitle
		\vspace{100mm}
	\end{mybox}
	\newpage
	\tableofcontents
	\listoffigures	
	\listoftables
	\newpage
	
%\chapter*{}	
%\begin{adjustwidth}{50pt}{50pt}
%		\textit{In the present document the design and model derivation of a CLL-Resonant DC/DC converter is reported. The document includes the control strategy and additional control loops for load harmonic compensation.}
%\end{adjustwidth}
\section{Introduction}
In this document the following topics are slightly covered: 
\begin{itemize}
	\item[--] phase locked loop for single phase signals;
	\item[--] moving average filter;
	\item[--] discrete Fourier transform;
	\item[--] Simulink C-caller;
	\item[--] dSPACE SCALEXIO.	
\end{itemize} 
The idea of the document is to propose an laboratory application of a single-phase \textit{PLL}, and moving average filter implemented in a fast prototyping equipment. The moving average filter will be implemented using customized C-code and Simulink C-caller.
\section{Nomenclature}
Here a list of symbols, variables, and parameters used along the document: 
	\begin{itemize}
		\item[--] \textit{PLL}: phase locked loop;
		\item[--] \textit{VCO}: voltage controlled oscillator;		
		\item[--] \textit{PI}: proportional integral controller;
		\item[--] \textit{LF}: loop filter;
		\item[--] \textit{PD}: phase detector;		
		\item[--] \textit{SOGI}: second order generalized integrator;
		\item[--] \textit{QSG}: quadrature signal generator;
		\item[--] \textit{MAVG}: moving average;
		\item[--] \text{RMB}: right mouse button;
		\item[--] \text{LMB}: left mouse button;
		\item[--] \text{HMI}: human machine interface;
		\item[--] \text{ConfigurationDesk}: dSPACE application used to configure a project with the scalexio equipment;
		\item[--] \text{ControlDesk}: dSPACE application used as HMI;
		\item[--] $v,\ v_{signal}\quad\Big[\SI{}{\volt}\Big]$: input signals;		
		\item[--] $\alpha\beta$: direct and quadrature components of a vector quantity, generally with respect to a stationary reference frame;		
		\item[--] $\xi\eta$: additional direct and quadrature components of a vector quantity, generally with respect to a rotating reference frame;
	\end{itemize}

\section{Basic Structure of a Phase-Locked Loop}
The basic structure of the phase-locked loop (\textit{PLL}) is shown in Figure~\ref{basic_pll_1}. It consists of three fundamental blocks:
\begin{itemize}
	\item[--] \textit{phase detector} (\textit{PD}). This block generates an output signal proportional to the phase difference between the input signal, $v_{signal}$, and the signal generated by the internal oscillator of the \textit{PLL}, $v^{pll}$. Depending on the type of \textit{PD}, high frequency AC components appear together with the DC phase-angle difference signal.
	\item[--] \textit{loop filter} (\textit{LF}). This block presents a low pass filtering characteristic to attenuate the high frequency AC components from the PD output. Typically, this block is constituted by a first order low pass filter or a PI controller.
	\item[--] \textit{voltage controlled oscillator} (\textit{VCO}). This block generates at its output an AC signal whose frequency is shifted with respect to a given central frequency, $\omega_{ff}$, as a function of the input voltage provided by the \textit{LF}.
\end{itemize}
\begin{figure}[H]
	\centering
	\includegraphics[width = 350pt, angle = 0, 
	keepaspectratio]{figures/basic_pll_1.eps}
	\captionsetup{width=0.5\textwidth, font=small}
	\caption{Basic structure of a PLL.}
	\label{basic_pll_1}
\end{figure}
The block diagram of an elementary \textit{PLL} is shown in Figure~\ref{basic_pll_2}. In this case \textit{PD} is implemented by means of the \textit{Superheterodyne} technique, the \textit{LF} is based on a \textit{PI} controller and the \textit{VCO} consists of a sinusoidal function supplied by a linear integrator.
\begin{figure}[H]
	\centering
	\includegraphics[width = 490pt, angle = 0, 
	keepaspectratio]{figures/basic_pll_2.eps}
	\captionsetup{width=0.5\textwidth, font=small}
	\caption{Block diagram of an elementary \textit{PLL}.}
	\label{basic_pll_2}
\end{figure}
If the input signal applied to this system is given by
\begin{equation}
	v=V\sin(\vartheta) = V\sin(\omega t + \phi)
\end{equation} 
and the signal generated by the \textit{VCO} is given by
\begin{equation}
	v^{pll}=\cos(\vartheta_{pll})=\cos(\omega_{pll}t+\phi_{pll})
\end{equation}
the phase error signal from the multiplier \textit{PD} output can be written as
\begin{equation}
	\begin{aligned}
		\varepsilon_{pd} &= Vk_{pd} \sin(\omega t +\phi)\cos(\omega_{pll}t+\phi_{pll}) \\[6pt]
		&= \frac{Vk_{pd}}{2}\Big[\sin[(\omega-\omega_{pll})t+(\phi-\phi_{pll})]+\sin[(\omega+\omega_{pll})t+(\phi+\phi_{pll})]\Big]
	\end{aligned}
\end{equation}
The high frequency components ($\omega+\omega_{pll}$) of the \textit{PD} error signal will be cancelled out by the \textit{LF}, only the low frequency term ($\omega-\omega_{pll}$) will be processed, therefore, the \textit{PD} error signal to be considered is
\begin{equation}
		\begin{aligned}
			\varepsilon_{pd} &=  \frac{Vk_{pd}}{2}\sin[(\omega-\omega_{pll})t+(\phi-\phi_{pll})]
		\end{aligned}
\end{equation}
If it is assumed that the \textit{VCO} is well tuned to the input frequency, i.e. with $\omega\approx\omega_{pll}$, the DC term of the phase error is given as follows
\begin{equation}
	\begin{aligned}\label{bpll_eq1}
		\varepsilon_{pd} &=  \frac{Vk_{pd}}{2}\sin(\phi-\phi_{pll})
	\end{aligned}
\end{equation}
It can be observed in \eqref{bpll_eq1} that the multiplier \textit{PD} produces nonlinear phase detection because of the sinusoidal function. However, when phase error is very small, i.e. when $\phi\approx\phi_{pll}$, the output of the multiplier \textit{PD} can be linearized in the vicinity of such an operating point since $\sin(\phi-\phi_{pll})\approx\sin(\vartheta-\vartheta_{pll})\approx(\vartheta-\vartheta_{pll})$. Therefore, once the \textit{PLL} is locked, the relevant term of the phase error signal is given by
 \begin{equation}
 	\begin{aligned}\label{bpll_eq2}
 		\varepsilon_{pd} &=  \frac{Vk_{pd}}{2}(\vartheta-\vartheta{pll})
 	\end{aligned}
 \end{equation}
According to Eq.~\eqref{bpll_eq2}, the model presented in Figure~\ref{basic_pll_2} can be linearized around the condition of $\omega\approx\omega_{pll}$ resulting as per Figure~\ref{basic_pll_3}.
\begin{figure}[H]
	\centering
	\includegraphics[width = 400pt, angle = 0, 
	keepaspectratio]{figures/basic_pll_3.eps}
	\captionsetup{width=0.5\textwidth, font=small}
	\caption{Small signal model of an elementary \textit{PLL}.}
	\label{basic_pll_3}
\end{figure}
According to the block diagram of Figure~\ref{basic_pll_3} a frequency domain analysis brings to the following transfer functions (consider $k_{pd}=k_{vco}=1$):
\begin{equation}\label{basic_pll_tf_eq1}
	H(s)=PD(s)LF(s)VCO(s)=\frac{k_ps+\frac{k_p}{\tau_i}}{s^2} 
\end{equation}
\begin{equation}\label{basic_pll_tf_eq2}
	H_{\vartheta}(s)=\frac{\Theta_{pll}(s)}{\Theta(s)}=\frac{H(s)}{1+H(s)}= \frac{k_ps+\frac{k_p}{\tau_i}}{s^2+k_ps+\frac{k_p}{\tau_i}}
\end{equation}
\begin{equation}\label{basic_pll_tf_eq3}
	E_{\vartheta}=\frac{E_{pd}(s)}{\Theta(s)}=1-H_{\vartheta}(s)=\frac{s^2}{s^2+k_ps+\frac{k_p}{\tau_i}}
\end{equation}
The open loop transfer function of Eq.~\eqref{basic_pll_tf_eq1} shows that the \textit{PLL} has two poles at the origin, which means that is able to track even a constant slope ramp in the input phase angle without any steady state error. 

 

\section{SOGI-QSG-based PLL}
Figure~\ref{vco_1} shows the structure of a \textit{VCO} based on \textit{QSG}; the structure consists of an adaptive filter.  
\begin{figure}[H]
	\centering
	\includegraphics[width = 400pt, angle = 0, keepaspectratio]{figures/vco_1.eps}
	\captionsetup{width=0.5\textwidth, font=small}
	\caption{Voltage controlled oscillator based on a adaptive filter.}
	\label{vco_1}
\end{figure}
Defining $g=k\epsilon_{v}$, the $v_\xi$, and $v_\eta$ components of Figure~\ref{vco_1} can be written as follows
\begin{equation}
	v_\xi=g\cos(\omega_{pll})=\frac{1}{2} g \Big(e^{j\omega_{pll}t}+e^{-j\omega_{pll}t}\Big)
\end{equation}
\begin{equation}
	v_\eta=g\cos(\omega_{pll})=-j\frac{1}{2} g \Big(e^{j\omega_{pll}t}-e^{-j\omega_{pll}t}\Big)
\end{equation}
The $\mathbf{A}_\xi$, and $\mathbf{A}_\eta$ terms which correspond to the output of the integrators for $v_\xi$ and $v_\eta$, can be expressed in the Laplace domain as follows
\begin{equation}
	\mathbf{A}_{\xi} = \frac{1}{s} v_\xi(s) = \frac{1}{2s}\Big[g(s+j\omega_{pll}t)+g(s-j\omega_{pll}t)\Big]
\end{equation}
\begin{equation}
	\mathbf{A}_{\eta} = \frac{1}{s} v_\eta(s) = -j\frac{1}{2s}\Big[g(s+j\omega_{pll}t)-g(s-j\omega_{pll}t)\Big]
\end{equation}
and the $v_\alpha^{pll}$, $v_\beta^{pll}$ terms result as follows
\begin{equation}
	\begin{aligned}
		v_\alpha^{pll} &=\frac{1}{2} \Big[\mathbf{A}_\xi(s+j\omega_{pll}t)+\mathbf{A}_\xi(s-j\omega_{pll}t)\Big] \\[6pt]
		&= \frac{1}{4(s+j\omega_{pll})}\Big[g(s)+g(s+j\omega_{pll}t)\Big]+\frac{1}{4(s-j\omega_{pll})}\Big[g(s)+g(s-j\omega_{pll}t)\Big]
	\end{aligned}
\end{equation}
\begin{equation}
	\begin{aligned}
		v_\beta^{pll} &= -j\frac{1}{2} \Big[\mathbf{A}_\xi(s+j\omega_{pll}t)-\mathbf{A}_\xi(s-j\omega_{pll}t)\Big] \\[6pt]
		&= \frac{1}{4(s+j\omega_{pll})}\Big[g(s)+g(s+2j\omega_{pll}t)\Big]+\frac{1}{4(s-j\omega_{pll})}\Big[g(s)-g(s-2j\omega_{pll}t)\Big]
	\end{aligned}
\end{equation}
The term $v^{pll}=v_\alpha^{pll}+v_\beta^{pll}$ results as follows
\begin{equation}\label{sogi_pll_eq1}
	\begin{aligned}
		v^{pll} = v_\alpha^{pll}+v_\beta^{pll} = \frac{s}{s^2+\omega_{pll}^2}g(s)
	\end{aligned}
\end{equation}
Consequently, the transfer functions of the adaptive filter \textit{VCO} structure of Figure~\ref{vco_1} are given by
\begin{equation}\label{sogi_pll_eq2}
	\frac{v^{pll}}{\varepsilon_{v}}(s)=\frac{ks}{s^2+\omega_{pll}^2}
\end{equation}
\begin{equation}
	\frac{v^{pll}}{v}(s)=\frac{ks}{s^2+ks+\omega_{pll}^2}
\end{equation}
\begin{equation}
	\frac{\varepsilon_{v}}{v}(s)=\frac{s^2+\omega_{pll}^2}{s^2+ks+\omega_{pll}^2}
\end{equation}
The structure of Figure~\ref{vco_1} can be used of quadrature signal generator (\textit{QSG}) by adding a scaled integrator at the output of the adaptive filter, as in Figure~\ref{qsg_1} is shown. 
\begin{figure}[H]
	\centering
	\includegraphics[width = 350pt, angle = 0, 
	keepaspectratio]{figures/qsg_1.eps}
	\captionsetup{width=0.5\textwidth, font=small}
	\caption{Quadrature signal generator based on a adaptive filter.}
	\label{qsg_1}
\end{figure}
Clearly, the response of the AF block, shown in Figure~\ref{vco_1} is defined by Eq.~\eqref{sogi_pll_eq2} in the case of applying a likewise sinusoidal signal (sine or cosine) with frequency $\omega_{pll}$ to its input.

Recalling that
\begin{equation}
	\mathscr{L}\Big[\sin(\omega_{pll})\Big]=\frac{\omega_{pll}}{s^2+\omega_{pll}^2}
\end{equation}
\begin{equation}
	\mathscr{L}\Big[\cos(\omega_{pll})\Big]=\frac{s}{s^2+\omega_{pll}^2}
\end{equation}
the time response of the system characterized by Eq.~\eqref{sogi_pll_eq2} in the presence of sinusoidal inputs is given by
\begin{equation}
	\mathscr{L}^{-1}\Bigg[\frac{\omega_{pll}}{s^2+\omega_{pll}^2}\frac{s}{s^2+\omega_{pll}^2}\Bigg]=\frac{1}{2}t\sin(\omega_{pll}t)
\end{equation}
and
\begin{equation}
	\mathscr{L}^{-1}\Bigg[\frac{s}{s^2+\omega_{pll}^2}\frac{s}{s^2+\omega_{pll}^2}\Bigg]=\frac{1}{2}\Bigg[\frac{\sin(\omega_{pll}t)}{\omega_{pll}}+t\cos(\omega_{pll}t)\Bigg]\approx\frac{1}{2}t\cos(\omega_{pll}t)
\end{equation}

An efficient implementation of the structure of Figure~\ref{qsg_1} is shown in Figure~\ref{sogi_1}, where
\begin{figure}[H]
	\centering
	\includegraphics[width = 325pt, angle = 0, 
	keepaspectratio]{figures/sogi_1.eps}
	\captionsetup{width=0.5\textwidth, font=small}	
	\caption{Second order adaptive filter based on an second order generalized integrator and a quadrature signal generator (\textit{SOGI-QSG}).}
	\label{sogi_1}
\end{figure}
\begin{equation}\label{sogi_pll_eq3}
	\textit{SOGI}(s)=\frac{v_\alpha^{pll}}{\varepsilon_{v}}(s)=\frac{k\,\omega_{pll}s}{s^2+\omega_{pll}^2}
\end{equation}
\begin{equation}\label{sogi_pll_eq4}
	\textit{D}(s)=\frac{v_\alpha^{pll}}{v_{}}(s)=\frac{k\,\omega_{pll}s}{s^2+k\,\omega_{pll}s+\omega_{pll}^2}
\end{equation}
\begin{equation}\label{sogi_pll_eq5}
	\textit{Q}(s)=\frac{v_\beta^{pll}}{v_{}}(s)=\frac{k\,\omega_{pll}^2}{s^2+k\,\omega_{pll}s+\omega_{pll}^2}
\end{equation}
Based on the structure of the \textit{SOGI-QSG} of Figure~\ref{sogi_1} it is possible to implement a \textit{SOGI-QSG}-based \textit{PLL} as shown in Figure~\ref{sogi_pll_1}.
\begin{figure}[H]
	\centering
	\includegraphics[width = 495pt, angle = 0, 
	keepaspectratio]{figures/sogi_pll_2.eps}
	\captionsetup{width=0.5\textwidth, font=small}	
	\caption{Diagram of the \textit{SOGI}-based \textit{PLL} (\textit{SOGI-QSG}).}
	\label{sogi_pll_1}
\end{figure}
The transfer function from the input signal $v$ to the error signal $\varepsilon_{v}$ is given by
\begin{equation}\label{sogi_pll_eq6}
	\textit{E}(s)=\frac{\varepsilon_{v}}{v}(s) = \frac{s^2+\omega_{pll}^2}{s^2+k\,\omega_{pll}s+\omega_{pll}^2}
\end{equation}
The transfer function of Eq.~\eqref{sogi_pll_eq6} responds to a second order notch filter, with zero gain at the centre frequency ($\omega_{pll}$).
\begin{figure}[H]
	\centering
	\begin{subfigure}{.5\textwidth}
		\centering
		\includegraphics[width = 200pt, angle = 0,	keepaspectratio]{figures/sogi_Q(s)_bode.eps}
		\captionsetup{width=0.75\textwidth, font=footnotesize}
		\caption{Bode diagram of the $\textit{Q}(s)$ transfer function in a \textit{SOGI-QSG}.}
		\label{bode_response_Q}
	\end{subfigure}%
	\begin{subfigure}{.5\textwidth}
		\centering
		\includegraphics[width = 200pt, angle = 0,	keepaspectratio]{figures/sogi_E(s)_bode.eps}
		\captionsetup{width=0.75\textwidth, font=footnotesize}
		\caption{Bode diagram of the $\textit{E}(s)$ transfer function in a \textit{SOGI-QSG}.}
		\label{bode_response_E}
	\end{subfigure}
	\captionsetup{width=0.5\textwidth, font=small}
	\caption{Bode diagram of the \textit{SOGI-QSG} response.}
	\label{}
\end{figure}
Another useful possible implementation of the \textit{SOGI-QSG}-based \textit{PLL} is based on the direct and inverse Park transforms, as shown in Figure~\ref{park_pll_1}
\begin{figure}[H]
	\centering
	\includegraphics[width = 495pt, angle = 0, keepaspectratio]{figures/single_phase_pll_fig1b.eps}
	\captionsetup{width=0.5\textwidth, font=small}	
	\caption{PLL based on the on direct and inverse Park transform.}
	\label{park_pll_1}
\end{figure}
\begin{multicols}{2}
An equivalent transfer function block diagram is presented in Figure~\ref{park_pll_2}. In this configuration the transformation $v_\alpha\rightarrow v_\beta^{pll}$ is represented as $\omega_{pll}/s$ in Laplace domain. An intuitive explanation of its operation principle can be given if it is assumed that the \textit{PLL} is well tuned to the input signal frequency. Under such operation conditions, if $v_\alpha$ and $v_\beta^{pll}$ are not in quadrature, the virtual input vector, $v_\alpha^{pll}$, resulting from these signals will have neither constant amplitude nor rotation speed.
	\begin{figure}[H]
	\centering
	\includegraphics[width = 235pt, angle = 0, keepaspectratio]{figures/qsg_park_eq_1.eps}
	\captionsetup{width=0.35\textwidth, font=small}	
	\caption{Equivalent block diagram of the PLL based on the on direct and inverse Park transform.}
	\label{park_pll_2}
\end{figure}
 Therefore $v_\xi$ and $v_\eta$ waveform resulting from direct Park transformation will have harmonics components. These harmonics will be suppressed by the \textit{LPF} blocks generating, as well, the components $v_\xi^f$ and $v_\eta^f$. The $v_\alpha$ and $v_\beta^{pll}$ components resulting from the inverse Park transformation of the components $v_\xi^f$ and $v_\eta^f$ will be in quadrature, though $v_\alpha$ and $v_\alpha^{pll}$ will not be in phase if the \textit{PLL} is not perfectly synchronized. As the \textit{PLL} locks the phase angle of the input signal $v_\alpha$, the components $v_\alpha^{pll}$ and $v_\beta^{pll}$ will be respectively in phase and in quadrature respect to the input signal $v_\alpha$.
\end{multicols}
From the equivalent block diagram of Figure~\ref{park_pll_2} the following transfer functions can be derived
\begin{flalign}
	\boxed{\frac{V_\beta^{pll}}{V_\alpha}(s)=\frac{\omega_f\omega_{pll}}{s^2+\omega_{f}s+\omega_{pll}^2}}
\end{flalign}
\begin{flalign}
	\boxed{\frac{V_\alpha^{pll}}{V_\alpha}(s)=\frac{s\omega_f}{s^2+\omega_{f}s+\omega_{pll}^2}}
\end{flalign}
where the relation of Eq.~\eqref{alphabeta_tr_eq} has been applied.
\begin{flalign}\label{alphabeta_tr_eq}
	V_\beta^{pll}(s) = \frac{\omega_{pll}}{s}V_\alpha^{pll}(s)
\end{flalign}
%\begin{equation}
%	V_\xi(s) = \frac{1}{2}\Big[\Big(V_\alpha(s+j\omega_{pll})+V_\alpha(s-j\omega_{pll})\Big)-j\Big(V_\beta^{pll}(s+j\omega_{pll})-V_\beta^{pll}(s-j\omega_{pll})\Big)\Big]
%\end{equation}
%\begin{equation}
%	V_\eta(s) = \frac{1}{2} \Big[ j\Big(V_\alpha(s+j\omega_{pll})-V_\alpha(s-j\omega_{pll})\Big)+ \Big(V_\beta^{pll}(s+j\omega_{pll})+V_\beta^{pll}(s-j\omega_{pll})\Big)\Big]
%\end{equation}
%
%\begin{equation}
%	V_\xi^{f}(s) = \frac{\omega_{f}}{s+\omega_{f}}V_\xi(s)
%\end{equation}
%\begin{equation}
%	V_\eta^{f}(s) = \frac{\omega_{f}}{s+\omega_{f}}V_\eta(s)
%\end{equation}
%
%\begin{equation}
%	V_\xi^f(s) = \frac{1}{2}\frac{\omega_{f}}{s+\omega_{f}}\Big[\Big(V_\alpha(s+j\omega_{pll})+V_\alpha(s-j\omega_{pll})\Big)-j\Big(V_\beta^{pll}(s+j\omega_{pll})-V_\beta^{pll}(s-j\omega_{pll})\Big)\Big]
%\end{equation}
%\begin{equation}
%	V_\eta^f(s) = \frac{1}{2}\frac{\omega_{f}}{s+\omega_{f}}\Big[j\Big(V_\alpha(s+j\omega_{pll})-V_\alpha(s-j\omega_{pll})\Big)+ \Big(V_\beta^{pll}(s+j\omega_{pll})+V_\beta^{pll}(s-j\omega_{pll})\Big)\Big]
%\end{equation}
%
%\begin{equation}
%	V_\alpha^{pll}(s) = \frac{1}{2}\Big[ \Big(V_\xi^f(s+j\omega_{pll})+V_\xi^f(s-j\omega_{pll})\Big) + j\Big(V_\eta^{f}(s+j\omega_{pll})-V_\eta^{f}(s-j\omega_{pll})\Big)\Big]
%\end{equation}
%\begin{equation}
%	V_\beta^{pll}(s) = \frac{1}{2}\Big[-j\Big(V_\xi^f(s+j\omega_{pll})-V_\xi^f(s-j\omega_{pll})\Big) + \Big(V_\eta^{f}(s+j\omega_{pll})+V_\eta^{f}(s-j\omega_{pll})\Big)\Big]
%\end{equation}

\begin{figure}[H]
	\centering
	\begin{subfigure}{.5\textwidth}
		\centering
		\includegraphics[width = 200pt, angle = 0,	keepaspectratio]{figures/Hin.eps}
		\captionsetup{width=0.75\textwidth, font=footnotesize}
		\caption{Bode diagram of the $\frac{V_\alpha^{pll}}{V_\alpha}(s)$ transfer function, with $\omega_f=2\pi\SI{50}{\hertz}$, and $\omega_{pll}=2\pi\SI{400}{\hertz}$.}
		\label{}
	\end{subfigure}%
	\begin{subfigure}{.5\textwidth}
		\centering
		\includegraphics[width = 200pt, angle = 0,	keepaspectratio]{figures/Hq.eps}
		\captionsetup{width=0.75\textwidth, font=footnotesize}
		\caption{Bode diagram of the $\frac{V_\beta^{pll}}{V_\alpha}(s)$ transfer function, with $\omega_f=2\pi\SI{50}{\hertz}$, and $\omega_{pll}=2\pi\SI{400}{\hertz}$.}
		\label{}
	\end{subfigure}
	\captionsetup{width=0.5\textwidth, font=small}
	\caption{Bode diagram of the \textit{QSG-PLL} based on direct and inverse Parck transform.}
	\label{}
\end{figure}
%\begin{equation*}
%	\begin{aligned}
%		V_\alpha^{pll}(s) &= \frac{1}{2}\Big\{\frac{1}{2}\frac{\omega_{f}}{s+\omega_{f}}\Big[\Big(V_\alpha(s+2j\omega_{pll})+V_\alpha(s)\Big)-j\Big(V_\beta^{pll}(s+2j\omega_{pll})-V_\beta^{pll}(s)\Big)\Big] + \\[6pt]
%	&\quad+\frac{1}{2}\frac{\omega_{f}}{s+\omega_{f}}\Big[\Big(V_\alpha(s)+V_\alpha(s-2j\omega_{pll})\Big)-j\Big(V_\beta^{pll}(s)-V_\beta^{pll}(s-2j\omega_{pll})\Big)\Big] + \\[6pt]
%	&\quad+j\frac{1}{2}\frac{\omega_{f}}{s+\omega_{f}}\Big[j\Big(V_\alpha(s+2j\omega_{pll})-V_\alpha(s)\Big)+ \Big(V_\beta^{pll}(s+2j\omega_{pll})+V_\beta^{pll}(s)\Big)\Big]+ \\[6pt]
%	&\quad-j\frac{1}{2}\frac{\omega_{f}}{s+\omega_{f}}\Big[j\Big(V_\alpha(s)-V_\alpha(s-2j\omega_{pll})\Big)+ \Big(V_\beta^{pll}(s)+V_\beta^{pll}(s-2j\omega_{pll})\Big)\Big]\Big\}
%	\end{aligned}
%\end{equation*}
%
%\begin{equation*}
%	\begin{aligned}
%		V_\beta^{pll}(s) &= \frac{1}{2}\Big\{-j\frac{1}{2}\frac{\omega_{f}}{s+\omega_{f}}\Big[\Big(V_\alpha(s+2j\omega_{pll})+V_\alpha(s)\Big)-j\Big(V_\beta^{pll}(s+2j\omega_{pll})-V_\beta^{pll}(s)\Big)\Big] + \\[6pt]
%		&\quad+j\frac{1}{2}\frac{\omega_{f}}{s+\omega_{f}}\Big[\Big(V_\alpha(s)+V_\alpha(s-2j\omega_{pll})\Big)-j\Big(V_\beta^{pll}(s)-V_\beta^{pll}(s-2j\omega_{pll})\Big)\Big] + \\[6pt]
%		&\quad+\frac{1}{2}\frac{\omega_{f}}{s+\omega_{f}}\Big[j\Big(V_\alpha(s+2j\omega_{pll})-V_\alpha(s)\Big)+ \Big(V_\beta^{pll}(s+2j\omega_{pll})+V_\beta^{pll}(s)\Big)\Big]+ \\[6pt]
%		&\quad+\frac{1}{2}\frac{\omega_{f}}{s+\omega_{f}}\Big[j\Big(V_\alpha(s)-V_\alpha(s-2j\omega_{pll})\Big)+ \Big(V_\beta^{pll}(s)+V_\beta^{pll}(s-2j\omega_{pll})\Big)\Big]\Big\}
%	\end{aligned}
%\end{equation*}




\begin{thebibliography}{99}
	% power electronics
	\bibitem{liserre} 
	R. Teodorescu, M. Liserre, P. Rodrıguez, \emph{Grid converters for photovoltaic and wind power systems}. Wiley, 2011.
\end{thebibliography}

\end{document} 